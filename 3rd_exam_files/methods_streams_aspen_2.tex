% Created 2022-12-09 Παρ 20:17
% Intended LaTeX compiler: pdflatex
\documentclass[11pt]{article}
\usepackage[utf8]{inputenc}
\usepackage[T1]{fontenc}
\usepackage{graphicx}
\usepackage{longtable}
\usepackage{wrapfig}
\usepackage{rotating}
\usepackage[normalem]{ulem}
\usepackage{amsmath}
\usepackage{amssymb}
\usepackage{capt-of}
\usepackage{hyperref}
\usepackage{booktabs}
\usepackage{import}
\usepackage[LGR, T1]{fontenc}
\usepackage[greek, english]{babel}
\usepackage{alphabeta}
\usepackage{esint}
\usepackage{mathtools}
\usepackage{esdiff}
\usepackage{makeidx}
\usepackage{glossaries}
\usepackage{newfloat}
\usepackage{minted}
\usepackage{chemfig}
\usepackage{svg}
\usepackage[a4paper, margin=3cm]{geometry}
\author{Βιδιάνος Γιαννίτσης, Διονύσης Γιαννάτος, Αριστοτέλης Αργυρόπουλος \\ Στυλιανή Σταύρου, Έλλη Πούτα, Θεωφανώ Πόταρη}
\date{\today}
\title{Ρεύματα, Διεργασίες, Υπολογιστικά και Θερμοδυναμικά Μοντέλα - Σχεδιασμός Διεργασιών - 3η πρόοδος}
\hypersetup{
 pdfauthor={Βιδιάνος Γιαννίτσης, Διονύσης Γιαννάτος, Αριστοτέλης Αργυρόπουλος \\ Στυλιανή Σταύρου, Έλλη Πούτα, Θεωφανώ Πόταρη},
 pdftitle={Ρεύματα, Διεργασίες, Υπολογιστικά και Θερμοδυναμικά Μοντέλα - Σχεδιασμός Διεργασιών - 3η πρόοδος},
 pdfkeywords={},
 pdfsubject={},
 pdfcreator={Emacs 28.2 (Org mode 9.5.5)}, 
 pdflang={English}}
\begin{document}

\maketitle
\tableofcontents

\renewcommand{\abstractname}{Περίληψη}
\renewcommand{\tablename}{Πίνακας}
\renewcommand{\figurename}{Σχήμα}
\renewcommand\listingscaption{Κώδικας}

\pagebreak

\section{Ολοκληρωμένο Διάγραμμα Ροής}
\label{sec:org85a2753}
Για την πρόοδο αυτή, έχουμε το ολοκληρωμένο διάγραμμα ροής της διεργασίας απεικονισμένο σχηματικά. Καθώς το διάγραμμα αυτό έγινε σε χαρτί Α3 με landscape configuration, θα πρέπει να γίνει πολύ zoom out για να μπορέσει να χωρέσει αυτό στο κείμενο αυτό το οποίο είναι Α4. Για αυτό, το διάγραμμα μπορεί να βρεθεί \href{https://github.com/Vidianos-Giannitsis/Process-Design/blob/master/Diagrams/complete\_flowsheet.pdf}{εδώ}. 

\section{Steam Explosion}
\label{sec:org646799b}
To steam explosion είναι η πρώτη διεργασία που χρησιμοποιείται με βάση το διάγραμμα ροής της διεργασίας. Σκοπός του είναι να μπορέσει με χρήση ατμού σε υψηλή πίεση και θερμοκρασία να κάνει ένα πρώτο διαχωρισμό της βιομάζας. Συγκεκριμένα, διώχνει το σημαντικότερο ποσοστό της ημικυτταρίνης η οποία είναι η πιό υδατοδιαλυτή και θερμοευαίσθητη μεταξύ αυτής, της κυτταρίνης και της λιγνίνης.

Το ρεύμα εισόδου είναι 200000 τόνοι πυρηνόξυλο το οποίο ορίστηκε ως ένα Nonconventional component και τα στοιχεία που απαιτούνται για αυτό (proximate \& ultimate analysis) βρέθηκαν βιβλιογραφικά \cite{koutsomitopoulouPreparationCharacterizationOlive2014,gonzalezCombustionOptimisationBiomass2004} 

Στην έξοδο υπάρχουν τρία ρεύματα. Το ένα ρεύμα έχει όλη την κυτταρίνη και τη λιγνίνη οι οποίες δεν διασπάστηκαν θερμικά. Οι ενώσεις αυτές ορίστηκαν στο Aspen ως Nonconventional compounds και τα δεδομένα που απαιτούνται βρέθηκαν από τη βάση δεδομένων \href{https://phyllis.nl/Browse/Standard/ECN-Phyllis\#}{Phyllis2}. Το δεύτερο είναι η διαλυμένη ημικυτταρίνη που έχει αυτουδρολυθεί στο περιβάλλον της έκρηξης ατμού παράγοντας ξυλόζη, άλλες πεντόζες και εξόζες (καθώς ήταν όλες μικρές ποσότητες, υπολογίστηκαν υποθέτοντας ότι όλες οι ζάχαρες που δεν είναι ξυλόζη στο ρεύμα αυτό είναι αραβινόζη), διάφορες φαινολικές ενώσεις από την διάσπαση της λιγνίνης (για τον ίδιο λόγο θεωρήθηκαν όλες φαινόλη). Η σύσταση του ρεύματος αυτού υπάρχει στην βιβλιογραφία \cite{fernandez-bolanosSteamexplosionOliveStones2001} . Το τρίτο ρεύμα είναι αέριο. Κατά την έκρηξη ατμού, υπάρχει απώλεια μάζας από την τροφοδοσία μέχρι τα δύο παραπάνω ρεύματα εξόδου. Αυτό συμβαίνει διότι η βιομάζα είναι θερμοευαίσθητη και ένα κομμάτι της διασπάται θερμικά. Για να μην γίνει ολοκληρωμένη ανάλυση του κλάσματος που διασπάστηκε, θεωρήθηκε πως όλη η αρχική υγρασία έγινε νερό (το οποίο με βάση την θερμοδυναμική κατανέμεται και στην υγρή και στην αέρια φάση σε ένα βαθμό), όλο το άζωτο της βιομάζας απελευθερώθηκε ως αέριο και οι υπόλοιπες απώλειες αποτελούν παραγωγή CO\textsubscript{2}.

Ο αντιδραστήρας του Steam Explosion ορίστηκε ως ένας RYield, καθώς δεν ορίζεται κάποια κινητική για την διεργασία αυτή, αλλά ξέρουμε βιβλιογραφικά τα ρεύματα εξόδου \cite{fernandez-bolanosSteamexplosionOliveStones2001,fernandez-bolanosCharacterizationLigninObtained1999}. Το θερμοδυναμικό μοντέλο που επιλέχθηκε είναι το μοντέλο SRK καθώς το σύστημα είναι ένα χημικό σύστημα σε υψηλή πίεση και ένα από τα μοντέλα που προτείνει το method assistant του Aspen για τέτοια συστήματα είναι το SRK.

Για περισσότερες πληροφορίες κοιτάξτε το αρχείο \href{https://github.com/Vidianos-Giannitsis/Process-Design/blob/master/Aspen/steam\_explosion.org}{αυτό} . Το σχετικό αρχείο του Aspen βρίσκεται \href{https://github.com/Vidianos-Giannitsis/Process-Design/blob/master/Aspen/steam\_explosion\_3phases.apwz}{εδώ}. Παρακάτω φαίνονται οι πληροφορίες αυτές και πινακοποιημένες.

\pagebreak

\begin{table}[htbp]
\caption{Πληροφορίες σχετικά με την διεργασία της έκρηξης ατμού}
\centering
\begin{tabular}{l}
\hline
Steam Explosion\\
\hline
Θερμοδυναμικό Μοντέλο - SRK\\
Μοντέλο Αντιδραστήρα - RYield\\
Ρεύματα:\\
\hline
Πυρηνόξυλο\\
\hline
Μη υδατοδιαλυτή φάση  (Λιγνίνη + Κυτταρίνη)\\
\hline
Υδατοδιαλυτή φάση. Ημικυτταρινικό\\
κλάσμα με κύριο συστατικό την ξυλόζη\\
\hline
Αέρια (CO\textsubscript{2}, N\textsubscript{2}, H\textsubscript{2O})\\
\hline
\end{tabular}
\end{table}

\subsection{Απορίες}
\label{sec:orgb358aa3}
\begin{enumerate}
\item Μας ενδιαφέρει να "καθαρίσουμε" το υδατοδιαλυτό ρεύμα ώστε να έχει μόνο ξυλόζη (από την οποία παράγεται η κυκλοπεντανόνη) ή δεν χρειάζεται να ασχοληθούμε.
\item Η προσομοίωση στον RYield τρέχει με Warning πως δεν ισχύουν τα ισοζύγια μάζας για C, H, N. Αυτό είναι πιθανόν να συμβαίνει λόγω συνδυασμού πολλών βιβλιογραφικών δεδομένων (επειδή δεν βρέθηκε άρθρο που να δίνει όλα τα απαιτούμενα δεδομένα). Λόγω αυτού, η προσομοίωση δίνει τα αναμενόμενα αποτελέσματα με το Warning αυτό. Πειράζει να μην ασχοληθούμε περαιτέρω με το warning αυτό καθώς δεν υπάρχει κάποιος εύκολος τρόπος να το ρυθμίσουμε;
\end{enumerate}

\section{Διαχωρισμός Κυτταρίνης και Λιγνίνης}
\label{sec:org3add0a8}
Η διεργασία αυτή αφορά τον διαχωρισμό κυτταρίνης και λιγνίνης μετά την έκρηξη ατμού καθώς είναι και οι δύο δυσδιάλυτες στον ατμό. Με βάση την βιβλιογραφία \cite{rodriguezOliveStoneAttractive2008}, μπορεί να γίνει μία αλκαλική εκχύλιση των στερεών, στην οποία, η λιγνίνη θα διαλυθεί ενώ η κυτταρίνη θα παραμείνει ως υπολειπόμενο στερεό. Για την εκχύλιση χρησιμοποιείται υδατικό
διάλυμα υδροξειδίου του νατρίου σε 50\% κατά βάρος. Αυτή η διεργασία είναι απαραίτητη για την ανάκτηση της λιγνίνης και για την καλύτερη λειτουργία των ενζύμων της υδρόλυσης της κυτταρίνης.

Ως είσοδος θεωρείται το κλάσμα λιγνίνης και κυτταρίνης που εξέρχεται από
τον αντιδραστήρα της έκρηξης ατμού και το υδατικό διάλυμα υδροξειδίου,
ενώ ή έξοδος αποτελείται από την καθαρισμένη κυτταρίνη και την υδατική φάση που έχει διάλυμα υδροξειδίου του νατρίου και διαλυτοποιημένη λιγνίνη.  

Για την μοντελοποίηση αυτής της διεργασίας, δεν βρέθηκε κάποια
αντίστοιχη διεργασία στο Aspen Plus, άρα χρησιμοποιήθηκε ο αντιδραστήρας
είδους Swash, ο οποίος μοντελοποιεί την διαδικασία της έκπλυσης στερεού,
και το θερμοδυναμικό μοντέλο που χρησιμοποιήθηκε ήταν το ELECNRTL, το
οποίο προτάθηκε από το Method Assistant του Aspen λόγω ύπαρξης
ηλεκτρολύτη στο νερό.

Διαπιστώθηκε πως στην έξοδο δεν υπήρχε καθόλου διαλυτοποιημένη λιγνίνη
στο ρεύμα εξόδου, το οποίο αποτελούταν εντελώς από νερό και υδροξείδιο
νατρίου. Είναι πιθανό το Aspen Plus, χωρίς δεδομένα για την
διαλυτοποίηση της λιγνίνης, να μην μπορεί να την προσομοιάσει, οδηγώντας
σε αυτήν την αστοχία.

Παρακάτω καταγράφονται συνοπτικά οι πληροφορίες αυτής της διεργασίας.

\begin{table}[htbp]
\caption{Πληροφορίες σχετικά με την διεργασία της αλκαλικής εκχύλισης}
\centering
\begin{tabular}{l}
\hline
Αλκαλική Εκχύλιση Λιγνίνης-Κυτταρίνης\\
\hline
Είδος Αντιδραστήρα - SWash\\
Θερμοδυναμικό Μοντέλο - ELECNRTL\\
Ρεύματα:\\
\hline
Είσοδος: Στερεή Λιγνίνη και Κυτταρίνη, Νερό (Ρεύμα 1)\\
Υδατικό Διάλυμα NaOH (Ρεύμα 2)\\
\hline
Έξοδος: Στερεή κυτταρίνη με μικρή ποσότητα λιγνίνης (Ρεύμα 1)\\
Υδατικό διάλυμα NaOH με διαλυτοποιημένη την\\
λιγνίνη της τροφοδοσίας (Ρεύμα 2)\\
\hline
\end{tabular}
\end{table}

\subsection{Απορίες}
\label{sec:org871f58b}
\begin{enumerate}
\item Καθώς η δοκιμή που έγινε για την διεργασία δεν πέτυχε, έχετε μήπως εσείς κάποια ιδέα για το πως να περάσουμε την διεργασία αυτή στο Aspen. Μία ιδέα μας είναι να βάλουμε μία κλασσική στήλη εκχύλισης (παρότι αυτή υποθέτει εκχύλιση υγρού-υγρού) και να δούμε αν θα τρέξει. Αλλά αυτό είναι πιθανόν να μην δουλέψει.
\item Στην περίπτωση που κάνουμε σωστά την προσομοίωση της εκχύλισης υγρού-στερεού, πιστεύουμε πως υπάρχει και πάλι πιθανότητα να μην δουλέψει καθώς η λιγνίνη έχει οριστεί ως non-conventional compound και μάλλον το Aspen δεν μπορεί να αναγνωρίσει ότι η λιγνίνη είναι διαλυτή στο καυστικό νάτριο. Έχετε κάποια ιδέα για το τι μπορούμε να κάνουμε;
\end{enumerate}

\section{Αντιδραστήρας Σακχαροποίησης}
\label{sec:org0aecfe0}
Ο αντιδραστήρας της ενζυμικής σακχαροποίησης αποτελεί την τρίτη
διεργασία στην σειρά, μετά τις διεργασίες έκρηξης ατμού και εκχύλισης με
διάλυμα υδροξειδίου του νατρίου για την αφαίρεση περίσσειας λιγνίνης.

Ως είσοδος θεωρείται το προϊόν της διεργασίας εκχύλισης, δηλαδή η
κυτταρίνη με την λιγνίνη και ένα μεγάλο ποσό νερού που θα χρησιμοποιηθεί
για να διαλύσει την γλυκόζη. Παράλληλα, το ρεύμα εισόδου θεωρείται
πως δεν περιέχει υδροξείδιο του νατρίου για την απλοποίηση των υπολογισμών. Βέβαια, καθώς η κυτταρινή δεν είναι διαλυτή στο NaOH, αυτή η παραδοχή είναι καλή, καθώς ελάχιστη ποσότητα διαλύτη δεν θα πάει στην φάση της λιγνίνης.

Η λιγνίνη και η κυτταρίνη ορίζονται ως Non-conventional solids στο
Aspen, και τα βιβλιογραφικά δεδομένα για τον ορισμό των ιδιοτήτων τους
βρέθηκαν από την \href{https://phyllis.nl/Browse/Standard/ECN-Phyllis}{βάση δεδομένων Phyllis}.

Στην έξοδο έχουμε ως υγρό την γλυκόζη που έχει προκύψει από την
σακχαροποίηση και την στερεή κυτταρίνη και λιγνίνη, που θα διαχωριστούν
από το υγρό και θα επιστρέψουν στον αντιδραστήρα για περεταίρω υδρόλυση.
Παράλληλα, εξέρχεται και μια μεγάλη ποσότητα νερού για να διαλύσει την
γλυκόζη.

Παρά το γεγονός πως έχει βρεθεί η κινητική της αντίδρασης ενζυμικής
σακχαροποίησης από την βιβλιογραφία \cite{kadamDevelopmentValidationKinetic2004}, το μοντέλο αυτό είναι αρκετά
περίπλοκο, και δεν ενδέχεται απλοποίηση και εισαγωγή στο Aspen εύκολα,
κυρίως λόγω της πολυμερικής φύσης της κυτταρίνης. Γι' αυτό το λόγο,
προτιμήθηκε η χρήση βιβλιογραφικών δεδομένων για την εύρεση της απόδοσης
της αντίδρασης, τουλάχιστον μέχρι να μπορέσει να απλοποιηθεί το μοντέλο
σε ικανοποιητικό βαθμό και να οριστεί κατάλληλα η κυτταρίνη. Η απόδοση
της αντίδρασης, σύμφωνα με την βιβλιογραφία, αποδίδει 87.7\% γλυκόζη από
την εισαγόμενη κυτταρίνη μετά από παραμονή 72 ωρών. Η λιγνίνη και το
νερό θεωρούνται ως αδρανή. Οι υπολογισμοί που έγιναν εκτός του Aspen φαίνονται \href{https://github.com/Vidianos-Giannitsis/Process-Design/blob/master/Calculations/Saccharification\_Calculations\_2.xlsx}{εδώ}.

Για αυτό τον λόγο, ο αντιδραστήρας ορίστηκε ως RYield, καθώς δεν απαιτεί
την εισαγωγή κάποιας κινητικής. Εφόσον το σύστημα λειτουργεί σε χαμηλή
πίεση (1 bar), χρησιμοποιήθηκε το θερμοδυναμικό μοντέλο NRTL, εφόσον
προτάθηκε και από το Method Assistant του Aspen.

Παρακάτω καταγράφονται συνοπτικά οι πληροφορίες αυτής της διεργασίας.

\begin{table}[htbp]
\caption{Πληροφορίες σχετικά με την διεργασία της ενζυμικής σακχαροποίησης}
\centering
\begin{tabular}{ll}
\hline
Διεργασία & Ενζυμική Σακχαροποίηση\\
\hline
Είδος Αντιδραστήρα & RYield\\
Θερμοδυναμικό Μοντέλο & NRTL\\
\hline
Ρεύματα: & Είσοδος: Στερεά (λιγνίνη, κυτταρίνη), Νερό\\
\hline
 & Έξοδος: Στερεά (λιγνίνη, κυτταρίνη), Νερό, Γλυκόζη\\
\hline
\end{tabular}
\end{table}

\section{Βιοαντιδραστήρας Παραγωγής Γλυκερόλης}
\label{sec:orgd4599fb}
Ο βιοαντιδραστήρας αυτός είναι μία από τις βασικές διεργασίες της εργασίας. Σκοπός του είναι να παράξει γλυκερόλη από γλυκόζη μέσω μικροοργανισμών. Επιλέχθηκε η χρήση του μικροοργανισμού C. glycerinogenes για την διεργασία αυτή και για αυτόν βρέθηκαν δύο βασικά πειράματα τα οποία βοήθησαν στην προσομοίωση του αντιδραστήρα \cite{zhugeGlycerolProductionNovel2001,jinByproductFormationNovel2003} . Αρχικά έγινε μία απλοποιημένη προσομοίωση όπου υποτέθηκε πως γλυκόζη και οξυγόνο δίνουν γλυκερόλη, διοξείδιο του άνθρακα και νερό και μόλις περάστηκε αυτή στο Aspen, δοκιμάστηκε η προσομοίωση της συνολικής αντίδρασης, όπου λαμβάνει υπόψην την βιομάζα, την πηγή αζώτου και τα παραπροιόντα.

Στην συνολική αυτή αντίδραση, τροφοδοτούμε τον αντιδραστήρα με υδατικό διάλυμα γλυκόζης και ουρίας δεδομένων συγκεντρώσεων καθώς και οξυγόνο. Στην βιβλιογραφία, αναφέρεται πως για την σωστή πραγματοποίηση της αντίδρασης, απαιτείται και κάποιο θρεπτικό μέσο όπως το Corn Steep Liquor. Αυτό είναι ένα "καλά ορισμένο" υγρό αλλά δεν υπάρχει στις βάσεις δεδομένων του Aspen. Ως αποτέλεσμα, πρέπει να περαστεί ως Conventional component όπου θα οριστούν από τον χρήστη όλες οι ιδιότητες του. Αυτό δημιουργεί προβλήματα επειδή κάποιες από τις ιδιότητες που ζητούνται δεν μπόρεσαν να βρεθούν και υποτέθηκαν ίσες με τις αντίστοιχες για το νερό. Η προσομοίωση αυτή έτρεξε με Warning ότι η αντίδραση έχει μη μηδενικό ρυθμό ένω έχει καταναλωθεί όλο το οξυγόνο (το οποίο είναι αντιδρών). Αυτό προκύπτει με το οξυγόνο που τροφοδοτείται για να τρέξει η προσομοίωση χωρίς CSL, το οποίο βρέθηκε αρκετό για να γίνει αντίδραση και να μην έχει πολύ περίσσεια. Σύμφωνα με το warning αυτό, για την αντίδραση με CSL θέλουμε περισσότερο οξυγόνο. Όμως, αν αλλάξει έστω και ελάχιστα η ποσότητα οξυγόνου, το warning αυτό γίνεται 3 errors. Για αυτό, το αρχείο complete\textsubscript{bioreactor} δεν το συμπεριλαμβάνει.

Τα προιόντα της αντίδρασης είναι γλυκερόλη (κύριο προιόν της ζύμωσης του C. glycerinogenes), μικροβιακή βιομάζα (η οποία αναπτύσσεται κατά την διάρκεια της αντίδρασης και την αυτοκαταλύει), νερό και διοξείδιο του άνθρακα (απαραίτητα προιόντα της μικροβιακής ζύμωσης) και αιθανόλη και οξικό οξύ τα οποία είναι τα παραπροιόντα της αντίδρασης \cite{zhugeGlycerolProductionNovel2001} . Βιβλιογραφικά παράγεται και αραβιτόλη, αλλά η προσθήκη της αραβιτόλης δημιουργούσε σοβαρά προβλήματα στην προσομοίωση του καθαρισμού της γλυκερόλης για αυτό αποφασίσαμε να αγνοηθεί. Η στοιχειομετρία της αντίδρασης προέκυψε με βάση βιβλιογραφικά δεδομένα για τα yields της αντίδρασης \cite{jinByproductFormationNovel2003} με βάση την μεθοδολογία που περιγράφεται \href{https://github.com/Vidianos-Giannitsis/Process-Design/blob/master/bioreactor\_stoichiometry.org}{εδώ} . Το δυσκολότερο κομμάτι της προσομοίωσης εδώ ήταν η προσθήκη της μικροβιακής βιομάζας στο Aspen. Με βάση τους \cite{wooleyDevelopmentASPENPhysical1996}, βρέθηκε μία τεχνική για να γίνει αυτό, η οποία περιγράφεται με περισσότερη λεπτομέρεια \href{https://github.com/Vidianos-Giannitsis/Process-Design/blob/master/biomass\_modeling\_aspen.org}{εδώ}.

Το υπολογιστικό μοντέλο που χρησιμοποιήθηκε για την προσομοίωση του βιοαντιδραστήρα είναι το RBatch καθώς στην βιβλιογραφία ο αντιδραστήρας αυτός είναι batch και υπάρχουν επαρκή δεδομένα για την προσομοίωση αυτή στο Aspen. Ο αντιδραστήρας θεωρήθηκε πως λειτουργεί σε σταθερή πίεση και θερμοκρασία μέχρι το ρεύμα εξόδου να έχει την επιθυμητή ποσότητα γλυκερόλης ή να περάσουν 80 ώρες (βιβλιογραφική διάρκεια αντίδρασης \cite{jinByproductFormationNovel2003} ). Για την κινητική της αντίδρασης, δεν υπάρχει διαθέσιμο στο Aspen το μοντέλο Monod το οποίο χρησιμοποιείται τυπικά για να περιγράψει την κινητική ανάπτυξης ενός μικροοργανισμού. Μπορεί όμως να προσομοιωθεί το μοντέλο αυτό ως LHHW με κατάλληλο ορισμό των παραμέτρων αυτού όπως φαίνεται στο \href{https://github.com/Vidianos-Giannitsis/Process-Design/blob/master/Aspen/simplified\_bioreactor.org}{αρχείο αυτό}.

Για τις θερμοδυναμικές παραμέτρους του προβλήματος χρησιμοποιήθηκε το μοντέλο NRTL-HOC το οποίο είναι κατάλληλο για χημικά συστήματα σε χαμηλή πίεση όπου υπάρχουν οργανικά οξέα. Περισσότερες πληροφορίες για την προσωμοίωση, υπάρχουν \href{https://github.com/Vidianos-Giannitsis/Process-Design/blob/master/Aspen/complete\_bioreactor.org}{εδώ}. Παρακάτω φαίνονται οι πληροφορίες αυτές και πινακοποιημένες

\begin{table}[htbp]
\caption{Πληροφορίες σχετικά με τον βιοαντιδραστήρα παραγωγής γλυκερόλης}
\centering
\begin{tabular}{l}
\hline
Βιοαντιδραστήρας Παραγωγής Γλυκερόλης\\
\hline
Θερμοδυναμικό Μοντέλο - NRTL-HOC\\
Μοντέλο Αντιδραστήρα - RBatch\\
Ρεύματα:\\
\hline
Υδατικό διάλυμα γλυκόζης και ουρίας + οξυγόνο\\
\hline
Υδατικό διάλυμα γλυκερόλης, βιομάζας, παραπροιόντων\\
και περισσευόμενων θρεπτικών μέσων\\
\hline
\end{tabular}
\end{table}

\subsection{Απορίες}
\label{sec:org7463bcf}
\begin{enumerate}
\item Το ρεύμα εξόδου από τον βιοαντιδραστήρα είναι περίπου 70\% νερό κατά μάζα. Για αυτό, σκεφτόμασταν μήπως αξίζει πριν τον καθαρισμό της γλυκερόλης από τα άλλα προιόντα της αντίδρασης να γίνει μία ξήρανση. Αρχικά, πως σας ακούγεται αυτό σαν ιδέα; Όμως, στο Model Palette του Aspen δεν βλέπω κάτι σαν ξηραντήρα άρα ήθελα να σας ρωτήσω και πως μπορούμε να προσομοιώσουμε την ξήρανση στο λογισμικό. Φαντάζομαι πως καθώς η ξήρανση είναι ένα φαινόμενο που έχει κινητική, η προσομοίωση θα γίνει σε έναν αντιδραστήρα, αλλά και πάλι δεν είμαι σίγουρος πως θα το κάναμε αυτό καθώς πως ακριβώς ορίζουμε την "στοιχειομετρία" για κάτι τέτοιο.
\end{enumerate}

\subsection{Σχόλια}
\label{sec:orgebdd35b}
Η προσομοίωση του βιοαντιδραστήρα δίνει χρόνο λειτουργίας πολύ μικρότερο του βιβλιογραφικού. Αυτό συμβαίνει λόγω παραδοχών που έγιναν κατά τους υπολογισμούς και συγκεκριμένα βασικό πρόβλημα είναι πως έχει υποτεθεί πως παράγεται πολύ περισσότερη βιομάζα από ότι παράγεται πραγματικά, το οποίο αυξάνει πάρα πολύ τον ρυθμό. Εν τέλει όμως, διαπιστώθηκε πως υπάρχουν τα δεδομένα για να βρεθεί ο στοιχειομετρικός συντελεστής της βιομάζας (δηλαδή η ποσότητα βιομάζας στην έξοδο του αντιδραστήρα). Αλλαγή της στοιχειομετρίας της αντίδρασης, θα προκαλέσει αλλαγή στον τύπο της βιομάζας ο οποίος παράγεται, με αποτέλεσμα να πρέπει να ξαναγίνουν αρκετοί υπολογισμοί. Λόγω χρόνου, η προσομοίωση θα διορθωθεί μετά την πρόοδο.

\section{Απομάκρυνση αζωτούχων από τα προϊόντα της βιοαντίδρασης}
\label{sec:orgae3ec3c}
Σύμφωνα με τους \cite{wallersteinMethodRecoveringGlycerol1946} , για την πιό αποτελεσματική απόσταξη των προιόντων της ζύμωσης, πρέπει πρώτα να απομακρυνθούν όλα τα αζωτούχα συστατικά στην έξοδο του βιοαντιδραστήρα. Για την διεργασία αυτή δεν βρέθηκαν άλλα δεδομένα, αλλά σύμφωνα με το παραπάνω, απαιτείται λιγνίνη η οποία μπορεί να δημιουργήσει σύμπλοκα με τα αζωτούχα συστατικά και μετά, με οξίνιση του διαλύματος, οι ενώσεις αυτές να δημιουργήσουν ίζημα. Η προσομοίωση της διεργασίας αυτής είναι ιδιαίτερα δύσκολη λόγω της έλλειψης αυτής δεδομένων.

Το ρεύμα εισόδου εδώ είναι όλα τα αζωτούχα συστατικά στην έξοδο (πλην της βιομάζας που είναι εξαρχής στερεή και μπορεί να απομακρυνθεί εύκολα). Αυτά είναι η υπολειπόμενη ουρία και οι πρωτείνες (χάριν ευκολίας μοντελοποιήθηκαν όλες ως αλανίνη, η οποία είναι η επικρατέστερη) και αμμωνία του CSL. Οι ποσότητες πάρθηκαν από την προσομοίωση της βιοαντίδρασης με το CSL παρόλο που αυτή τρέχει με ένα warning.

Το ρεύμα εξόδου θεωρείται πως είναι ένα nonconventional υλικό με τη σύσταση της λιγνίνης αν προστεθεί στη δομή της η κάθε αζωτούχος ένωση. Η διαδικασία των υπολογισμών αυτών περιγράφεται \href{https://github.com/Vidianos-Giannitsis/Process-Design/blob/master/Aspen/bioreactor\_nitrogen\_removal.org}{εδώ}.

Ο αντιδραστήρας που χρησιμοποιήθηκε είναι ένας RStoic λόγω των ελάχιστων δεδομένων που υπάρχουν για την αντίδραση. Στο αρχείο που έγιναν οι προαναφερόμενοι υπολογισμοί, έγινε μία προσπάθεια να προκύψει και μία στοιχειομετρία για την αντίδραση, αλλά με βάση τα δεδομένα που μπορούν να περαστούν στο Aspen αυτή δεν έβγαζε σωστά αποτελέσματα. Εν τέλει, η στοιχειομετρία που περάστηκε, περάστηκε μόνο επειδή έβγαζε το αναμενόμενο αποτέλεσμα (παράγεται σύμπλοκο της αζωτούχος ένωσης και της λιγνίνης με μάζα 2 φορές αυτήν της αζωτούχου ένωσης). Το θερμοδυναμικό μοντέλο που χρησιμοποιήθηκε είναι το NRTL. Ακολουθεί και πινακοποιημένη μορφή της προσομοίωσης όπως και παραπάνω

\begin{table}[htbp]
\caption{Πληροφορίες σχετικά με την απομάκρυνση αζωτούχων}
\centering
\begin{tabular}{l}
\hline
Απομάκρυνση Αζωτούχων από τον Αντιδραστήρα\\
\hline
Θερμοδυναμικό Μοντέλο - NRTL\\
Μοντέλο Αντιδραστήρα - RStoic\\
Ρεύματα:\\
\hline
Ουρία, αλανίνη και αμμωνία που περίσσεψαν από τον βιοαντιδραστήρα\\
Λιγνίνη\\
\hline
Σύμπλοκα αζωτούχων και Λιγνίνης\\
\hline
\end{tabular}
\end{table}

\subsection{Απορίες}
\label{sec:orgd395b45}
\begin{enumerate}
\item Λόγω των ελάχιστων δεδομένων που υπάρχουν για την αντίδραση, όπως θεωρώ έγινε κατανοητό, η προσομοίωση δεν ήταν ιδιαίτερα ακριβής ή ολοκληρωμένη. Θεωρείται αξίζει να την λάβουμε υπόψην στο τελικό διάγραμμα ροής, ή δεν αξίζει τον κόπο;
\end{enumerate}

\section{Καθαρισμός Γλυκερόλης}
\label{sec:org46be34a}
Μετά την διήθηση και την απομάκρυνση των αζωτούχων, υπάρχουν στο ρεύμα αιθανόλη, οξικό οξύ και γλυκερόλη διαλυμένα σε νερό. Η αιθανόλη και το οξικό οξύ είναι πολύ πτητικές ενώσεις σε σχέση με τη γλυκερόλη για αυτό μπορούν να απομακρυνθούν εύκολα με ένα flash. To flash καταφέρνει να διαχωρίσει το ρεύμα και στον πυθμένα υπάρχει διάλυμα γλυκερόλης-νερού σε σύσταση 74-26. Για να γίνει αυτό, η τροφοδοσία μπήκε στους 150 \(^oC\) και το flash λειτουργούσε σε θερμοκρασία 140 \(^oC\) και πίεση 1 atm.

Για τον διαχωρισμό αυτών χρησιμοποιήθηκε μία αποστακτική στήλη. Η προσομοίωση έγινε αρχικά σε dstwu κολώνα και έπειτα σε radfrac με στόχο το προιόν πυθμένα να είναι 0.9999 γλυκερόλη. Τα αποτελέσματα της προσομοίωσης έδειξαν πως η στήλη μπορεί να λειτουργήσει στις ίδιες συνθήκες με το flash, με πτώση πίεσης κατά μήκους της στήλης 0.1 atm (0.95 atm στην κορυφή και 1.05 atm στον πυθμένα). Οι βαθμίδες της στήλης είναι 4 ενώ η τροφοδοσία μπαίνει πάνω από την δεύτερη. Τέλος, o λόγος αναρροής στην στήλη είναι R = 0.16.

Παρακάτω παρατίθενται πινακοποιημένες πληροφορίες για την προσομοίωση.

\begin{table}[htbp]
\caption{Πληροφορίες σχετικά με τον καθαρισμό της γλυκερόλης}
\centering
\begin{tabular}{l}
\hline
Καθαρισμός Γλυκερόλης\\
\hline
Θερμοδυναμικό Μοντέλο - NRTL-HOC\\
Μοντέλα Διαχωριστήρων - Flash και Radfrac\\
Ρεύματα:\\
\hline
Υδατικό διάλυμα οξικού οξέος, αιθανόλης και γλυκερόλης\\
\hline
Καθαρή γλυκερόλη, νερό και υδατικό διάλυμα οξικού οξέος\\
και αιθανόλης\\
\hline
\end{tabular}
\end{table}

\section{Αντιδραστήρας Ξυλόζης}
\label{sec:orgfaee438}
\begin{center}
\includegraphics[width=.9\linewidth]{Αντιδραστήρας_Ξυλόζης/2022-12-09_20-15-47_screenshot.png}
\end{center}

Ο αντιδραστήρας αυτός είναι ο βασικός αντιδραστήρας που επεξεργάζεται το υδατικό ρεύμα στην έξοδο του steam explosion το οποίο έχει ως βασικό συστατικό την ξυλόζη. Τα δεδομένα φυσικών ιδιοτήτων όλων των στοιχείων ελήφθησαν από τη
βάση δεδομένων του Aspen. Το θερμοδυναμικό μοντέλο ΝRTL χρησιμοποιήθηκε για τον υπολογισμό των συντελεστών δραστηριότητας
υγρού, ενώ η εξίσωση κατάστασης Hayden-O'Connell (HOC) χρησιμοποιήθηκε
για τον υπολογισμό του θερμοδυναμικές ιδιότητες της φάσης ατμού. Επιπλέον, η
UNIFAC χρησιμοποιήθηκε για την εκτίμηση των δυαδικών παραμέτρων
στα δεδομένα ισορροπίας ατμού-υγρού (VLE) καθώς και σε όλες τις δυαδικές παραμέτρους στα δεδομένα ισορροπίας υγρού-υγρού για τη
διαδικασία εκχύλισης. Για το σχεδιασμό και τη βελτιστοποίηση της στήλης
απόσταξης έγιναν αρκετές προσομοιώσεις για να
προσδιοριστεί η αρχική δομή.

\subsection{Αντιδραστήρας για την παραγωγή φουρφουράλης}
\label{sec:orgab91be3}
Η αντίδραση για την παραγωγή φουρφουράλης είναι ομογενής κατάλυση και
αποτελεί την πρώτη διεργασία για την παραγωγή κυκλοπεντανόνης. Για την
είσοδο του αντιδραστήρα ως τροφοδοσία θεωρείται το ημικυτταρινικό κλάσμα
με κύριο συστατικό την ξυλόζη. Η έξοδος του αντιδραστήρα είναι πλούσια
σε φουρφουράλη. Επίσης στο ρεύμα εισόδου αλλά και στο ρεύμα εξόδου
περιέχεται νερό. Από την βιβλιογραφία η κινητική σταθερά της αντίδρασης
(k) ισούται με \(3.67 \cdot 10^9 \exp (\frac{-1.1 \cdot 10^9}{RT})\) \cite{nhienNovelHybridReactive2021} και η απόδοση σε
φουρφουράλη είναι 99\%. Ο αντιδραστήρας που χρησιμοποιήθηκε στο aspen
είναι ο αντιδραστήρας CSTR σε συνθήκες 243 \^{}\{ο\} C και 15.6 atm.

\begin{table}[htbp]
\caption{Πληροφορίες σχετικά με την αντίδρασης διάσπασης ξυλόζης σε φουρφουράλη και νερό}
\centering
\begin{tabular}{ll}
Είδος Αντιδραστήρα & RCSTR\\
\hline
Θερμοδυναμικό Μοντέλο & NRTL-HOC\\
Ρεύματα: & Είσοδος: Ξυλόζη, Νερό (Feed)\\
 & Έξοδος: Φουρφουράλη, Νερό (Stream 1)\\
\end{tabular}
\end{table}

\section{Καθαρισμός Φουρφουράλης}
\label{sec:org57364cb}
Το ακάθαρτο ρεύμα φουρφουράλης από την στήλη RD (αντιδραστικής
απόσταξης) ψύχεται στους 40 °C και πίεση 1 atm και εισάγεται στην κορυφή
ενός εκχυλιστήρα. Στον πυθμένα του εισάγεται ο διαλύτης βουτυλοχλωρίδιο.
Ο λόγος μάζας του ακάθαρτου ρεύματος φουρφουράλης προς τον διαλύτη
βρέθηκε από βιβλιογραφία ίσος με 8,11 \cite{nhienNovelHybridReactive2021} έτσι ώστε να επιτευχθεί η βέλτιστη
καθαρότητά της. Στο aspen plus για αυτή την διεργασία έγινε χρήση
Extractor.

\begin{table}[htbp]
\caption{Πληροφορίες σχετικά με την διεργασία εκχύλισης για διαχωρισμό φουρφουράλης από νερό}
\centering
\begin{tabular}{l}
Διεργασία Εκχύλισης Φουρφουράλης\\
\hline
Μοντέλο Εκχυλιστήρα: Extract\\
Θερμοδυναμικό Μοντέλο: UNIFAC\\
\hline
Ρεύματα:\\
\hline
Φουρφουράλη, Νερό (Stream 2)\\
Βουτυλοχλωρίδιο\\
\hline
Ρεύμα πλούσιο σε φουρφουράλη και βουτυλοχλωρίδιο (Stream Rich-Sol)\\
Νερό\\
\hline
\end{tabular}
\end{table}

Στη συνέχεια το ρεύμα που είναι πλούσιο σε φουρφουράλη οδηγείται σε έναν
αποστακτήρα και διαχωρίζεται έτσι ώστε το προϊόν πυθμένα να είναι
πλούσιο και καθαρό σε φουρφουράλη. Η αποστακτική στήλη που επιλέχτηκε
είναι DSTWU διότι αποτελείται από μία τροφοδοσία και δύο εξόδους, όπως
χρειάζεται στην περίπτωση μας. Από την κορυφή μπορεί να παραληφθεί ο
διαλύτης και να επαναχρησιμοποιηθεί με ανακύκλωση στην εκχύλιση και από
τον πυθμένα, μπορεί να παραληφθεί φουρφουράλη καθαρότητας 99\%. Όταν
ολοκληρωθούν οι υπολογισμοί μπορεί να υπολογιστεί το μικρότερο δυνατό
reflux ratio που απαιτείται, καθώς και τα ελάχιστα θεωρητικά στάδια της
απόσταξης.

\begin{table}[htbp]
\caption{Πληροφορίες σχετικά με την διεργασία απόσταξης για ανάκτηση καθαρής φουρφουράλης}
\centering
\begin{tabular}{ll}
Είδος Αποστακτήρα & DSTWU\\
\hline
Θερμοδυναμικό Μοντέλο & NRTL\\
Ρεύματα: & Είσοδος: Φουρφουράλη, Βουτυλοχλωρίδιο (Rich-Sol)\\
 & Έξοδος: Φουρφουράλη (Stream 4)\\
 & \\
 & Βουτυλοχλωρίδιο (Stream 3)\\
\end{tabular}
\end{table}

Το ρεύμα που εισάγεται στην αποστακτική στήλη με το τέλος της διεργασίας
επιστέφει μέσω ανακύκλωσης για να χρησιμοποιηθεί ξανά ο διαλύτης.

\begin{table}[htbp]
\caption{Πληροφορίες σχετικά με την διεργασία ανακύκλωσης του διαλύτη βουτυλοχλωρίδιο για χρησιμοποιήση του στην εκχύλιση}
\centering
\begin{tabular}{ll}
Ανακύκλωση & Mixer\\
\hline
Θερμοδυναμικό Μοντέλο & NRTL\\
Ρεύματα: & Είσοδος: Make-up, Stream 3\\
 & Έξοδος: Βουτυλοχλωρίδιο (Solvent)\\
\end{tabular}
\end{table}

\subsection{Απορίες}
\label{sec:org3f5cd72}
\begin{enumerate}
\item Η τροφοδοσία του αντιδραστήρα είναι η τιμή της ημικυτταρίνης που
βρέθηκε από το steam explosion, παρόλα αυτά κατά την εκτέλεση της
διεργασίας ο αντιδραστήρας αντιμετωπίζει πρόβλημα με το ισοζύγιο
μάζας με αποτέλεσμα να εμποδίζεται η πραγματοποίηση των υπόλοιπων
διεργασιών.

\item Δεν είναι γνωστό από την βιβλιογραφία ο χρόνος παραμονής και ο όγκος
του αντιδραστήρα.

\item Θέλουμε να χρησιμοποιήσουμε καταλύτη H\textsubscript{2}SO\textsubscript{4} αλλά δεν υπάρχει
δυνατότητα προσδιορισμού του στο aspen.

\item Η εκχύλιση έβγαζε την ένδειξη warring λόγω διαφοράς πίεσης στα
ρεύματα εισόδου και εξόδου και στον εκχυλιστήρα, από την βιβλιογραφία
η πίεση βρέθηκε ότι είναι ίση με 1 atm και στα 2 ρεύματα.
\end{enumerate}

\section{Παραγωγή κυκλοπεντανόνης από φουρφουράλη}
\label{sec:orga595858}

\begin{center}
\includegraphics[width=.9\linewidth]{Παραγωγή_κυκλοπεντανόνης_από_φουρφουράλη/2022-12-09_20-17-29_screenshot.png}
\end{center}

Αυτός είναι ο δεύτερος αντιδραστήρας που απαιτείται για την επεξεργασία του ρεύματος της ξυλόζης, ο οποίος έχει ως σκοπό την μετατροπή του ενδιάμεσου προιόντος (φουρφουράλη) στο τελικό επιθυμητό προιόν, την κυκλοπεντανόνη. Για την παραγωγή κυκλοπεντανόνης, αρχικά το ρεύμα που είναι πλούσιο σε
φουρφουράλη τροφοδοτείται σε αντιδραστήρα CSTR. Από την βιβλιογραφία \cite{yuHighlySelectiveHydrogenative2022} η
σταθερά της αντίδρασης για 160 \^{}\{ο\} C είναι ίση με 0,128 hr\textsuperscript{-1} με
ενέργεια ενεργοποίησης 64,2 kJ/mol. Ο χρόνος που χρειάζεται η αντίδραση
για να πραγματοποιηθεί σε αυτές τις συνθήκες είναι 1 ώρα. Στο aspen
έγινε χρήση του αντιδραστήρα RCSTR με το θερμοδυναμικό μοντέλο Wilson.

\begin{table}[htbp]
\caption{Πληροφορίες σχετικά με την αντίδραση διάσπασης φουρφουράλης}
\centering
\begin{tabular}{l}
\hline
Είδος Αντιδραστήρα - RCSTR\\
Θερμοδυναμικό Μοντέλο - Wilson\\
Ρεύματα:\\
\hline
Φουρφουράλη (Stream 1)\\
\hline
Κυκλοπεντανόνη και παραπροιόντα:\\
Κυκλοπεντανόλη, φουρφουραλική αλκοόλη\\
τετραϋδροφουρφουρυλική αλκοόλη και 5-υδρόξυ-πεντανόνη\\
\hline
\end{tabular}
\end{table}

Ακολούθως, το ρεύμα εξόδου του αντιδραστήρα εισάγεται σε αποστακτική
στήλη για τον διαχωρισμό της κυκλοπεντανόνης από τα υπόλοιπα προιόντα της αντίδρασης. Για τον λόγο αυτό στο aspen χρησιμοποιήθηκε
η αποστακτική στήλη RadFrac με το θερμοδυναμικό μοντέλο UNIFAC διότι
αυτή η στήλη μπορεί να πραγματοποιήσει πιο εκτενείς υπολογισμούς με
πολλαπλές ενώσεις. Στην περίπτωσή μας, τα προϊόντα της αντίδρασης στον
αντιδραστήρα CSTR είναι πέντε, οπότε η τροφοδοσία της αποστακτικής
στήλης έχει πέντε διαφορετικές ενώσεις. Η κυκλοπεντανόνη έχει χαμηλότερο
σημείο βρασμού από τα άλλα προιόντα, οπότε, στο τέλος της απόσταξης
μπορούμε να παραλάβουμε καθαρή κυκλοπεντανόνη από την κορυφή της στήλης,
και ένα ρεύμα με τις υπόλοιπες τέσσερις ενώσεις από τον πυθμένα. Τέλος ο
αριθμός των αριθμός των δίσκων απόσταξής είναι 10.

\begin{table}[htbp]
\caption{Πληροφορίες σχετικά με την διεργασία απόσταξης για διαχωρισμό των προιόντων της αντίδρασης και παραλαβή καθαρής κυκλοπεντανόνης}
\centering
\begin{tabular}{l}
\hline
Είδος κολώνας απόσταξης - Radfrac\\
Θερμοδυναμικό Μοντέλο - UNIFAC\\
Ρεύματα:\\
\hline
Κυκλοπεντανόνη και παραπροιόντα:\\
Κυκλοπεντανόλη, φουρφουραλική αλκοόλη\\
τετραϋδροφουρφουρυλική αλκοόλη και 5-υδρόξυ-πεντανόνη\\
\hline
Κυκλοπεντανόνη - Ατμώδης φάση\\
Παραπροιόντα - Υγρή φάση\\
\hline
\end{tabular}
\end{table}

\section{Γενικές απορίες}
\label{sec:org32f2aa6}
\begin{enumerate}
\item Προσομοιώνοντας ξεχωριστά την κάθε διεργασία έχουμε επιλέξει διαφορετικά μοντέλα για την κάθε αντίδραση ανάλογα με τις συνθήκες στις οποίες διεξάγονται και τις ενώσεις που παίρνουν μέρος. Έχει χρησιμοποιηθεί SRK για τα συστήματα υψηλής πίεσης (πχ Steam Explosion), NRTL-HOC για τα συστήματα όπου υπάρχει οργανικό οξύ (βιοαντιδραστήρας και καθαρισμός της γλυκερόλης) και NRTL για τα υπόλοιπα. Είδαμε στο UI του Aspen πως υπάρχει τρόπος να αλλάξουμε το θερμοδυναμικό μοντέλο για τις επιμέρους διεργασίες. Θεωρείται θα δημιουργήσει πρόβλημα αυτό όταν προσπαθήσουμε να ενώσουμε όλες τις διεργασίες;
\end{enumerate}

\section{Βιβλιογραφία}
\label{sec:org1617477}
\bibliography{../../../../../../Sync/My_Library}
\bibliographystyle{unsrt}
\end{document}