\documentclass[a4paper,12pt]{article}\usepackage[LGR, T1]{fontenc}\usepackage[greek,english]{babel}\usepackage{alphabeta}\usepackage{hyperref}\usepackage{chemformula}\usepackage{graphicx}\graphicspath{ {./PhotoLatex/} }\begin{document}\title{Σημειώσεις}\pagenumbering{roman}\tableofcontents\newpage\pagenumbering{arabic}\section{Start} 

Για τον διαχωρισμό των προϊόντων που αποτελούνται από αιθανόλη,γλυκερόλη οξικό οξύ όλα διαλυμένα σε νερό, χρησιμοποιήθηκε ένας flash διαχωρηστήρας που καταφέρνει να διαχωρίσει στο ρεύμα του πυθμένα γλυκερόλη και νερό(74-26\%). Τέλος για τον διαχωρισμό της γλυεκρόλης από το νερό, χρησιμοποιήθηκε ένα dstwu για να βρεθούν τα χαρακτηριστικά της αποστακτικής στήλης που θα χρειαστεί για τον διαχωρισμό. Αφού βρέθηκαν αυτά, εφαρμόστηκαν σε μια στήλη radfrac από την οποία επιτυγχάνεται ανάκτηση γλυκερόλης 0,9999\% στο ρεύμα του πυθμένα.


\subsection{Συνθήκες λειτουργείας στον flash}

\begin{itemize}
\item Θερμοκρασία εισόδου ρεύματος: 150 $^{o} C$
\item Θερμοκρασία λειτουργείας: 140 $^{o} C$
\item Πίεση λειουργείας: 1 atm
\end{itemize}

\subsection{Συνθήκες λειτουργείας στον radfrac}
\begin{itemize}
\item Θερμοκρασία εισόδου ρεύματος: 140 $^{o} C$
\item Θερμοκρασία λειτουργείας: 140$^{o} C$
\item Πίεση κορυφής: 0,95 atm
\item Πίεση πυθμένα:1,05 atm
\item Βαθμίδες: 4
\item Βαθμίδα τροφοδοσίας: 2
\item Λόγος αναρροής: 0,160034


\end{itemize}







 \end{document}